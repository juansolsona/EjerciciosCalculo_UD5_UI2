\documentclass[a4paper,11pt]{article}

\usepackage[spanish]{babel}
\usepackage[utf8]{inputenc}
\usepackage[T1]{fontenc}

\usepackage{amsmath, amssymb, amsthm}
\newtheorem{exercise}{Ejercicio}

\usepackage{geometry}
\geometry{margin=2.5cm}

\title{Ejercicios de Cálculo UD5}
\author{}
\date{}

\begin{document}

\maketitle
\begin{exercise}[Demostrar las fórmulas utilizadas para el gradiente cuando se usa la función sigmoide (1 punto).]


    
\[
\frac{\partial e}{\partial w_i} = (\hat y - y)\,\hat y\,(1 - \hat y)\,x_i
\]

\[
\frac{\partial e}{\partial b} = (\hat y - y)\,\hat y\,(1 - \hat y)
\]


\bigskip
\textbf{Solución}

\subsection*{1.1 Definamos primero cada elemento}

\subsubsection*{1.1.1 Modelo lineal}
\[
z = \sum_{i=1}^{n} w_i\,x_i + b
\]

\subsubsection*{1.1.2 Función sigmoide}
\[
\hat y = \sigma(z) = \frac{1}{1 + e^{-z}}
\]

\subsubsection*{1.1.3 Función de error}
\[
e(\hat y) = \frac{1}{2}\,(\hat y - y)^2
\]

\bigskip

\subsection*{1.2 Primeras derivadas}

\subsubsection*{1.2.1 Derivada del modelo de datos}

\[
\frac{\partial z}{\partial w_i}
= \lim_{h\to 0}\frac{z(w_i+h) - z(w_i)}{h}
\]

\textbf{Cada término para calcular el límite es:}

\begin{align*}
z(w_i+h) &= \sum_{j\neq i} w_j x_j + (w_i + h)x_i + b \\
z(w_i)   &= \sum_{j\neq i} w_j x_j + w_i x_i + b
\end{align*}

\textbf{Con lo que la expresión completa es:}

\[
\frac{z(w_i+h) - z(w_i)}{h}
=
\frac{\left[\sum_{j\neq i} w_j x_j + (w_i+h)x_i + b\right]
-\left[\sum_{j\neq i} w_j x_j + w_i x_i + b\right]}{h}
\]

\textbf{Simplificando:}

\[
\frac{h x_i}{h} = x_i
\]

\textbf{Ahora el límite es:}

\[
\lim_{h\to 0} x_i = x_i
\]

\textbf{Por lo que:}

\[
\frac{\partial z}{\partial w_i} = x_i.
\]

\bigskip

\subsubsection*{1.2.2 Derivada de la función sigmoide}

\[
f(z) = \frac{1}{1 + e^{-z}}
\]

\textbf{Definición de derivada:}

\[
f'(z) = \lim_{h \to 0}\frac{f(z+h) - f(z)}{h}
\]

\[
f(z) = \frac{1}{1+e^{-z}}, 
\quad
f(z+h) = \frac{1}{1 + e^{-(z+h)}}
\]

\[
f'(z) = \lim_{h \to 0} 
\frac{
\displaystyle \frac{1}{1 + e^{-(z+h)}} - \frac{1}{1 + e^{-z}}
}{h}
\]


\[
f'(z) = \lim_{h \to 0}
\frac{
(1+e^{-z}) - (1+e^{-(z+h)})
}{
h (1+e^{-(z+h)}) (1+e^{-z})
}
\]

\[
f'(z) = \lim_{h \to 0}
\frac{e^{-z} - e^{-(z+h)}}
{h (1+e^{-(z+h)}) (1+e^{-z})}
\]

\textbf{Por tanto:}

\[
f'(z) = \frac{e^{-z}}{(1+e^{-z})^2} = \hat y(1-\hat y)
\]

\[
\frac{\partial \hat y}{\partial z} = \hat y\,(1-\hat y)
\]

\bigskip

\subsubsection*{1.2.3 Derivada de la función de error}

\[
e(\hat y) = \frac{1}{2}(\hat y - y)^2
\]

\textbf{Definición de derivada:}

\[
\frac{\partial e}{\partial \hat y}
=
\lim_{h \to 0}
\frac{
\frac{1}{2}(\hat y + h - y)^2
-
\frac{1}{2}(\hat y - y)^2
}{h}
\]

\[
=
\frac{1}{2}
\lim_{h \to 0}
\frac{
(\hat y - y + h)^2 - (\hat y - y)^2
}{h}
\]

\textbf{Expansión del cuadrado:}

\[
(\hat y - y + h)^2
= (\hat y - y)^2 + 2h(\hat y - y) + h^2
\]

\[
\frac{\partial e}{\partial \hat y}
=
\frac{1}{2}
\lim_{h \to 0}
\frac{
2h(\hat y - y) + h^2
}{h}
\]

\textbf{Simplificando:}

\[
\frac{\partial e}{\partial \hat y}
=
\frac{1}{2}
\lim_{h \to 0}
\left[ 2(\hat y - y) + h \right]
\]

\[
\frac{\partial e}{\partial \hat y}
= \frac{1}{2} \cdot 2(\hat y - y)
= \hat y - y
\]

Ahora podemos aplicar la regla de la cadena:
Las derivadas que nos quedan son:
\[
\frac{\partial z}{\partial w_i}=x_i
\]
\[\frac{\partial \hat y}{\partial z}=\hat y(1-\hat y)\]
\paragraph{y}
\[\frac{\partial e}{\partial \hat y}=\hat y-y\]

Sabemos que se trata de una composicion de funciones:
\[e = e(\hat y(z(w_i)))\]

Por la regla de la cadena:
\[\frac{\partial e}{\partial w_i}=\frac{\partial e}{\partial \hat y}\cdot \frac{\partial \hat y}{\partial z} \cdot
\frac{\partial z}{\partial w_i}\]

Sustituyendo :
\[\frac{\partial e}{\partial \hat y} = \hat y - y\]
\[\frac{\partial \hat y}{\partial z} = \hat y(1-\hat y)\]
\[\frac{\partial z}{\partial w_i} = x_i\]

\textbf{Entonces:}

\[
\boldsymbol{
\frac{\partial e}{\partial w_i}= (\hat y - y)\,\hat y(1-\hat y)\,x_i
}
\]

Para $b$:
\[\frac{\partial e}{\partial b}=\frac{\partial e}{\partial \hat y}\cdot \frac{\partial \hat y}{\partial z} \cdot \frac{\partial z}{\partial b}\]

Pero, hay que tener en cuenta que:
\[\frac{\partial z}{\partial b} = 1\]

Por lo tanto:

\[\frac{\partial e}{\partial b}=(\hat y - y)\cdot \hat y(1-\hat y)\cdot 1\]

\textbf{O lo que es lo mismo::}

\[
\boldsymbol{
\frac{\partial e}{\partial b}=(\hat y - y)\,\hat y(1-\hat y)}\]
\end{exercise}

\end{document}